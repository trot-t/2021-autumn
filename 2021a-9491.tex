\providecommand{\pdfxopts}{a-1b,cyrxmp}
%\providecommand{\pdfxopts}{a-1b}
\providecommand{\thisyear}{2021}
\immediate\write18{rm \jobname.xmpdata}%  uncomment for Unix-based systems
\begin{filecontents*}{\jobname.xmpdata}
\Title{Ведомость датчики 9491, осень\textemdash\thisyear} % год поставится сам
\Author{Прокшин Артем Николаевич}
\Creator{pdfTeX + pdfx.sty with options \pdfxopts }
\Subject{Ведомость посещения занятий и выполнения лабораторных работ по датчикам студентами 9491 группы, осень 2021}
\Keywords{ведомость посещения, группа 9491, ЛЭТИ}
\CoverDisplayDate{март \thisyear}
\CoverDate{2021-09-04}
\Copyrighted{True}
\Copyright{Public Domain}
\CopyrightURL{http://github.com/trot-t}
\Creator{pdfTeX + pdfx.sty with options \pdfxopts }
\end{filecontents*}


\documentclass[a4paper,landscape,11pt]{article}

\pdfcompresslevel=9

\usepackage[\pdfxopts]{pdfx}[2016/03/09]
\PassOptionsToPackage{obeyspaces}{url}
\let\tldocrussian=1  % for live4ht.cfg



%\documentclass[a4paper,11pt]{article} 
%\usepackage[T1,T2A]{fontenc}
\usepackage[utf8x]{inputenc}
\usepackage[english,russian]{babel} 
\usepackage{wrapfig}
%\usepackage[table,xcdraw]{xcolor}
\usepackage{booktabs}
\usepackage{pifont}
\usepackage{graphicx}
\graphicspath{ {images/} }

\usepackage{tikz}
\usepackage{siunitx}
\usepackage[american,cuteinductors,smartlabels]{circuitikz}

\usepackage{hyperref}

\usepackage{advdate}
%\usepackage{showframe} % для отладки позиции на странице
\usepackage{cancel}

%\setlength{\voffset}{-72pt} %отступ сверху - чтобы увидеть откомментарить \usepackage{showframe}
\setlength{\voffset}{-56pt} %landscape
\setlength{\topmargin}{0pt} 
%\setlength{\headheight{1pt}
\setlength{\headsep}{0pt}
\setlength{\hoffset}{-222pt} %landscape
\setlength{\marginparwidth}{0pt}
\setlength{\textwidth}{800pt} %landscape
\setlength{\textheight}{538pt} %landscape
\setlength{\footskip}{-60pt}


%\author{ Прокшин Артем \\
%\small ЛЭТИ\\
%\small \texttt{taybola@gmail.com}}

%\date{}
%abcdefghijklmnop


\newcommand*\OK{&\small \ding{51}$\!\!_\circ$} % начал защищать
\newcommand*\Ok{&\small \ding{51}$\!\!_\circ$} % начал защищатi
\newcommand*\ok{&{\small \ding{51}}} % присутствовал
\newcommand*\oK{&{\small \ding{51}?}} % присутствовал?
\newcommand*\no{&{\small }} % отсутствовал
\newcommand*\D{\tiny\ding{48}} % защита, defend
\newcommand*\da{&{\small\ding{48}$\!\!_1$}} % защита, defend
\newcommand*\dab{&{\small\ding{48}$\!\!^1_2$}} % защита, defend
\newcommand*\ab{&{\small\ding{48}$\!\!^1_2$}} % защита, defend
\newcommand*\ad{&{\small${}^1\!\!$\ding{48}$\!\!_4$}} % защита, defend
%\newcommand*\ab{&{\small\ding{48}$\!\!^1_2$}} % защита, defend
\newcommand*\bc{&{\small\ding{48}$\!\!^2_3$}} % защита, defend
\newcommand*\dabc{&{\small\ding{48}$\!\!^1_{23}$}} % защита, defend
\newcommand*\dabcd{&{\small\ding{48}$\!\!^{12}_{34}$}} % защита, defend
\newcommand*\ac{&{\small\ding{48}$\!\!^1_{23}$}} % защита, defend
\newcommand*\db{&{\small\ding{48}$\!\!_2$}} % защита, defend
\newcommand*\dc{&{\small\ding{48}$\!\!_3$}} % защита, defend
\newcommand*\dd{&{\small\ding{48}$\!\!_4$}} % защита, defend
\newcommand*\bd{&{\small${}^2\!\!$\ding{48}$\!\!^3_{4}$}} % защита, defend
\newcommand*\de{&{\small\ding{48}$\!\!_5$}} % защита, defend
\newcommand*\dE{&{\small${}^4\!\!\!$\ding{48}$\!\!_5$}} % защита, defend
\newcommand*\cd{&{\small\ding{48}$\!\!^3_4$}} % защита, defend
\newcommand*\dg{&{\small\ding{48}$\!\!_6$}} % защита, defend
\newcommand*\fg{&{\small${}^6\!\!$\ding{48}$\!\!_7$}} % защита, defend
\newcommand*\dH{&{\small\ding{48}$\!\!_8$}} % защита, defend
\newcommand*\gh{&{\small\ding{48}$\!\!^7_8$}} % защита, defend
\newcommand*\fh{&{\small\ding{48}$\!\!^7_{89}$}} % защита, defend 
\newcommand*\ce{&{\small${}^3\!\!$\ding{48}$\!\!_5$}} % защита, defend
\newcommand*\ef{&{\small${}^5\!\!$\ding{48}$\!\!_6$}} % защита, defend
%\newcommand*\dh{&{\small\ding{48}$\!\!_8$}} % защита, defend
\newcommand*\di{&{\small\ding{48}$\!\!_9$}} % защита, defend
\newcommand*\cdef{&{\small ${}^2_4\!\!$\ding{48}$\!\!^{3}_{5}$}} % защита, defend
\newcommand*\cde{&{\small ${}^2\!\!$\ding{48}$\!\!^{3}_{5}$}} % защита, defend
\newcommand*\efg{&{\small ${}^5\!\!$\ding{48}$\!\!^{6}_{7}$}} % защита, defend
\newcommand*\befgh{&{\small ${}_2^5\!\!$\ding{48}$\!\!^{6}_{78}$}} % защита, defend
\newcommand*\Dh{&{\small${}^4\!\!$\ding{48}$\!\!_8$}} % защита, defend
\newcommand*\cfg{&{\small ${}^3\!\!$\ding{48}$\!\!^{6}_{7}$}} % защита, defend
\newcommand*\fgh{&{\small ${}^6\!\!$\ding{48}$\!\!^{7}_{8}$}} % защита, defend
\newcommand*\bce{&{\small ${}^2\!\!$\ding{48}$\!\!^{3}_{5}$}} % защита, defend
\newcommand*\dO{&{\small\ding{48}$\!\!_{15}$}}
\newcommand*\Skip{\noindent\rule{0.3cm}{0.9pt}}


\begin{document}
%\thispagestyle{empty}
% or
\pagenumbering{gobble}
%\AdvanceDate[-1] % печатаю в субботу а нужна пятница
%\begin{center}\today\end{center} % не до даты с 30 студентами
\vspace*{1\baselineskip} %landscape
\vspace{-0.9cm}
%\begin{table} \centering 
\newcommand*{\CS}{9pt} % ширина колонки
\begin{tabular}{p{7pt}|l|p{\CS}|p{\CS}|p{\CS}|p{\CS}|p{\CS}|p{\CS}|p{\CS}|p{\CS}|p{\CS}|p{\CS}}
%\multicolumn{16}{c}{График выполнения лабораторных работ студентами 8871 группы} \\ 
\multicolumn{11}{c}{Ведомость посещения занятий по датчикам студентами 9491 группы} \\
\toprule 
&&&&&&&&&&\\
&&&&&&&&&&\\
&&&&&&&&&&\\
&&&&&&&&&&\\
&&&&&&&&&&\\
&&&&&&&&&&\\
&&\rotatebox{90}{\rlap{\small 4 сентября}}
&\rotatebox{90}{\rlap{\small 18 сентября}}
&\rotatebox{90}{\rlap{\small 2 октября }}
&\rotatebox{90}{\rlap{\small 16 октября }}
&\rotatebox{90}{\rlap{\small 30 октября }}
&\rotatebox{90}{\rlap{\small 10 ноября/лекция }}
&\rotatebox{90}{\rlap{\small 13 ноября }}
&\rotatebox{90}{\rlap{\small 27 ноября }}
&\rotatebox{90}{\rlap{\small 11 декабряя }}
%&\rotatebox{90}{\rlap{\small 25 декабря }}
%&\rotatebox{90}{\rlap{\small 15 мая}}
%&\rotatebox{90}{\rlap{\small 29 мая }} 
\\

% commands vi to copy/paste D :+19 ->>> p :-18 :w
\midrule
1\,& Арнаутов Милан Сергеевич         \ok\no\ok\no\ok\no\no\no\no\\ 
2\,& Белкин Андрей Михайлович         \ok\ok\ok\ok\ok\no\ok\ok\ok\\ 
3\,& Богомолова Мария Владимировна    \no\no\no\no\ok\no\ok\ok\no\\ 
4\,& Виноградов Иван Алексеевич       \ok\ok\no\ok\ok\no\ok\ok\ok\\ 
5\,& Вэй Чжиюань                      \ok\ok\ok\ok\no\no\ok\ok\ok\\ 
\midrule                                          
6\,& Глушенкова Татьяна Вячеславовна  \ok\ok\ok\no\ok\no\ok\ok\ok\\ 
7\,& Горобец Андрей Андреевич         \no\ok\ok\ok\no\no\ok\no\ok\\ 
8\,& Жалейко Никита Денисович         \no\ok\ok\ok\ok\no\ok\no\no\\ 
9\,& Захаров Дмитрий Валерьевич       \ok\ok\ok\ok\ok\ok\ok\no\ok\\ 
10\,& Зубкова Валерия                 \ok\ok\ok\ok\ok\ok\ok\ok\ok\\
\midrule                                          
11\,& Клоков Андрей Андреевич         \ok\ok\ok\ok\ok\no\ok\ok\no\\ 
12\,& Кузьмин Андрей Александрович    \ok\no\ok\ok\no\no\ok\ok\ok\\ 
13\,& Кустов Данила Игоревич          \ok\ok\ok\ok\ok\no\ok\ok\ok\\
14\,& Ли Роллан                       \ok\ok\ok\ok\ok\no\ok\ok\ok\\
15\,& Литош Кирилл                    \ok\ok\ok\ok\ok\no\ok\ok\ok\\
\midrule
16\,& Лобазев Никита Александрович    \ok\ok\ok\ok\ok\no\ok\ok\ok\\
17\,& Мамяко Арина Игоревна           \ok\ok\ok\ok\no\no\ok\ok\ok\\
18\,& Масинович Артем Александрович   \ok\ok\ok\ok\ok\no\ok\ok\ok\\
19\,& Матвейчук Владислав             \ok\ok\ok\ok\ok\ok\ok\ok\ok\\
20\,& Матюшечкин Павел Андреевич      \ok\ok\ok\ok\ok\no\ok\no\no\\
\midrule
21\,& Миронова Анастасия Алексеевна   \ok\no\ok\ok\ok\no\ok\ok\ok\\
22\,& Пеев Валерий Романович          \ok\ok\ok\ok\ok\ok\ok\ok\ok\\
23\,& Петров Петр                     \ok\ok\ok\ok\ok\ok\ok\ok\ok\\
24\,& Салихов Карим Фаильевич         \ok\ok\no\ok\ok\no\ok\ok\ok\\
25\,& Соколов Михаил Олегович         \ok\ok\ok\ok\ok\no\ok\ok\no\\
\midrule
26\,& Чабан Олег Сергеевич            \ok\ok\ok\ok\no\ok\ok\ok\ok\\
27\,& Шамова Полина Витальевна        \ok\no\ok\ok\ok\no\ok\no\no\\
28\,& Шанина Анастасия Александровна  \ok\ok\ok\ok\ok\no\ok\ok\ok\\
29\,& Ягуткина Анастасия Владимировна \ok\ok\ok\ok\ok\no\ok\ok\ok\\
\bottomrule
\end{tabular} 

\newpage
%


\newpage
%
\begin{tabular}{l|llccccccccccccc}
\multicolumn{10}{c}{выполнение лабораторнах работ по датчикам, 9491 группа} \\
\toprule
&&Л1&Л1& Л2&Л2& Л3&Л3& Л4&Л4 &Л5&Л5& Л6&Л6\\
\midrule
1\,& Арнаутов Милан Сергеевич         \\
2\,& Белкин Андрей Михайлович         &    &      &15.10&16.10 &     &      &30.10&\\
3\,& Богомолова Мария Владимировна    &14.12&16.12& & &14.12&16.12\\
4\,& Виноградов Иван Алексеевич       &     &16.12&     &      &     &      &30.10&--&    &   &11.12&11.12\\
5\,& Вэй Чжиюань                      &.    &27.11\\
\midrule
6\,& Глушенкова Татьяна Вячеславовна  &17.09&18.09&15.10&16.10 &     &\\
7\,& Горобец Андрей Андреевич         &16.12&16.12&\\
8\,& Жалейко Никита Денисович         &     &16.12&     &      &     &      &30.10&     &13.11&13.11\\
9\,& Захаров Дмитрий Валерьевич       &     &     &15.10&16.10 &     &      &30.10&30.10\\
10\,& Зубкова Валерия                 &17.09&18.09&09.10&16.10 &23.10&27.11 &29.10&30.10&13.11&13.11&11.12&11.12\\
\midrule
11\,& Клоков Андрей Андреевич         &     &16.12&     &      &     &      &30.10&11.12\\
12\,& Кузьмин Андрей Александрович    &     &16.12&     &      &25.11 &25.11&     & & & &11.11&11.11\\
13\,& Кустов Данила Игоревич          &     &     &15.10&16.10 &15.10&27.11 &30.10&\\
14\,& Ли Роллан                       &     &16.12&01.12& 16.12&     &      &30.10&&&& 11.12&11.12\\
15\,& Литош Кирилл                    &24.09&11.12&15.10&16.10 &16.10&27.11 &11.12&30.10&13.11&13.11&11.12&11.12\\
\midrule
16\,& Лобазев Никита Александрович    &     &     &09.10&16.10 &     &      &30.10&     &12.11&13.11\\
17\,& Мамяко Арина Игоревна           &24.09&11.12&     &      &30.10&27.11 &03.11&11.12\\
18\,& Масинович Артем Александрович   &24.09&16.12&     &      &     &      &30.10&\\
19\,& Матвейчук Владислав             &22.09&16.10&15.10&16.10 &15.10&27.11 &29.10&30.10&13.11&13.11&11.11&11.11\\
20\,& Матюшечкин Павел Андреевич      \\
\midrule
21\,& Миронова Анастасия Алексеевна   &     &     &     &      &30.10&27.11&\\
22\,& Пеев Валерий Романович          &17.09&16.12&1.12 &16.12 &     &      &30.10&30.10\\
23\,& Петров Петр                     &18.09&18.09& 9.10&16.10 &23.10&27.11 &29.10&30.10&13.11&13.11&11.12&11.12\\
24\,& Салихов Карим Фаильевич         &     &     &     &      &     &      &30.10&     &13.11&13.11\\
25\,& Соколов Михаил Олегович         &24.09&16.12&     &      &     &      &30.10&\\
\midrule
26\,& Чабан Олег Сергеевич            &18.09&16.10&09.10&16.10&22.10 &27.11 &20.10&11.12&13.11&13.11&3.12&16.12\\
27\,& Шамова Полина Витальевна        &    &      &15.10&16.10 &     &      &30.10&\\
28\,& Шанина Анастасия Александровна  &14.12&16.12&15.10&16.10 &15.10 & 16.12&04.11&16.12&13.12&13.12&16.12&16.12\\
29\,& Ягуткина Анастасия Владимировна &18.09&18.09&15.10&16.10 &15.10 & 16.12&30.10&04.11&16.12&13.11&13.11&16.12\\
\bottomrule
\end{tabular}


\subsection*{Операционный усилитель}
\begin{itemize}
\item Глушенкова -- желательно графики
\item Чабан О.С. должна быть модель библиотеки. график 5 неверен, посчитайте по нему коэф усиления - увидите что ошиблись, весь график по аси ординат лежит в районе 14в.
\item Пеев  -- приложить исходные коды, без низх не принимаю.
\item ЛитошБ Виноградов, Матвейчук -- для напряжения семщения увеличены резисторы, но в этом случае на измерены $K_U$.
\end{itemize}

\subsection*{Избирательные усилители}
\begin{itemize}
\item Миронова -- идеальный вместо реального
\item Кустов, Мамяко -- нет темы, R не по ГОСТ, желательно логарифмический масштаб по оси y
\item Кузьмин идеальный ОУ.
\end{itemize}

\subsection*{Измерение напряжения сети}
\begin{itemize}
\item Петров -- Нет исходного кода, мне нечем проверить работу. Рисунок 4 -- график напряжений - нет обозначений величин на осях абсциссы и ординаты, нет цены деления осей.
В самом деле, осциллограф показывает мгновенное значение. Обычно для систем управления именно оно использеется. Для технических ссылок используется действующее.
В промышленной синусоидальной сети напряжение дается действующим значением,
а в преобразованном по амплитуде нужно войти в заданный диапазон мгновенных значений. По графику для $VM2$ видно, что не вошли. 
\item Матвейчук  -- в принциаиальных смехах Т-образные соединения проводников не по ГОСТу, точка должна быть в точке пересечения. Номер варианта не совпадает с опурационным усилителем.
должен быть LM111 (промахнудся на 1). Графики должны быть графиками а не снимками экрана осциллографа, поэтому не видна цена деления по осям.

\item Зубкова -- Плюс за описание индуктивно связанных катушек, есть еще индуктивность рассеяния обмоток, её эмулируют небольшими индуктивностями не связанными.
Рис. 4. Переходные процессы - нет обозначений величин на осях абсциссы и ординаты, нет цены деления осей.
Надо конкретно формулировать подпись к рисункам, иначе  к фразе "Переходные процессы" много вопросов.

%\item Починков -- напряжение на выходе должно занимать по возможности весь диапазон, за исключением резерва, от 0 до 3.3В. В работе $1.20-1.80$ вольт

\item Захаров -- первый, отметивший что для синусоидальной сети обычно задаеться действующее напряжение

\item Селихов -- изменить цифровой след (заголовок, тема, автор, ключевые слова) в соответствии с реальным авторством. Автором в документе стоит Петров. нет исходного кода.

\item Пеев -- в исходном коде стоит идеальный ОУ, а не ваш по заданию
\texttt{ XOP2         1 12 13 14 11StdOpamp}. По заданию должен быоь OPA411
\item  Ли Роллан -- принципиальная электрическая схема (ПЭ3) настолько смазана, что буквенно-позиционные обозначения (БПО) нельзя прочесть. Графики тоже в муаре. 
Напряжение на выходе должно занимать по возможности весь диапазон, за исключением резерва, от 0 до 3.3В. В работе $1.30-1.60$ вольт



\item Ягуткина, Шанина --
 выходное напряжение занимает область знячений в пределах 0.01в (по графику), а должна щт 0 до 3в, фильтр С в схеме не соответствует вычислению
\end{itemize}
\end{document}
