\providecommand{\pdfxopts}{a-1b,cyrxmp}
%\providecommand{\pdfxopts}{a-1b}
\providecommand{\thisyear}{2021}
\immediate\write18{rm \jobname.xmpdata}%  uncomment for Unix-based systems
\begin{filecontents*}{\jobname.xmpdata}
\Title{Ведомость датчики 9493, осень\textemdash\thisyear} % год поставится сам
\Author{Прокшин Артем Николаевич}
\Creator{pdfTeX + pdfx.sty with options \pdfxopts }
\Subject{Ведомость посещения занятий и выполнения лабораторных работ по датчикам студентами 9493 группы, осень 2021}
\Keywords{ведомость посещения, группа 9493, ЛЭТИ}
\CoverDisplayDate{март \thisyear}
\CoverDate{2021-09-04}
\Copyrighted{True}
\Copyright{Public Domain}
\CopyrightURL{http://github.com/trot-t}
\Creator{pdfTeX + pdfx.sty with options \pdfxopts }
\end{filecontents*}


\documentclass[a4paper,landscape,11pt]{article}

\pdfcompresslevel=9

\usepackage[\pdfxopts]{pdfx}[2016/03/09]
\PassOptionsToPackage{obeyspaces}{url}
\let\tldocrussian=1  % for live4ht.cfg



%\documentclass[a4paper,11pt]{article} 
%\usepackage[T1,T2A]{fontenc}
\usepackage[utf8x]{inputenc}
\usepackage[english,russian]{babel} 
\usepackage{wrapfig}
%\usepackage[table,xcdraw]{xcolor}
\usepackage{booktabs}
\usepackage{pifont}
\usepackage{graphicx}
\graphicspath{ {images/} }

\usepackage{tikz}
\usepackage{siunitx}
\usepackage[american,cuteinductors,smartlabels]{circuitikz}

\usepackage{hyperref}

\usepackage{advdate}
%\usepackage{showframe} % для отладки позиции на странице
\usepackage{cancel}

%\setlength{\voffset}{-72pt} %отступ сверху - чтобы увидеть откомментарить \usepackage{showframe}
\setlength{\voffset}{-56pt} %landscape
\setlength{\topmargin}{0pt} 
%\setlength{\headheight{1pt}
\setlength{\headsep}{0pt}
\setlength{\hoffset}{-222pt} %landscape
\setlength{\marginparwidth}{0pt}
\setlength{\textwidth}{800pt} %landscape
\setlength{\textheight}{538pt} %landscape
\setlength{\footskip}{-60pt}


%\author{ Прокшин Артем \\
%\small ЛЭТИ\\
%\small \texttt{taybola@gmail.com}}

%\date{}
%abcdefghijklmnop


\newcommand*\OK{&\small \ding{51}$\!\!_\circ$} % начал защищать
\newcommand*\Ok{&\small \ding{51}$\!\!_\circ$} % начал защищатi
\newcommand*\ok{&{\small \ding{51}}} % присутствовал
\newcommand*\oK{&{\small \ding{51}?}} % присутствовал?
\newcommand*\no{&{\small }} % отсутствовал
\newcommand*\D{\tiny\ding{48}} % защита, defend
\newcommand*\da{&{\small\ding{48}$\!\!_1$}} % защита, defend
\newcommand*\dab{&{\small\ding{48}$\!\!^1_2$}} % защита, defend
\newcommand*\ab{&{\small\ding{48}$\!\!^1_2$}} % защита, defend
\newcommand*\ad{&{\small${}^1\!\!$\ding{48}$\!\!_4$}} % защита, defend
%\newcommand*\ab{&{\small\ding{48}$\!\!^1_2$}} % защита, defend
\newcommand*\bc{&{\small\ding{48}$\!\!^2_3$}} % защита, defend
\newcommand*\dabc{&{\small\ding{48}$\!\!^1_{23}$}} % защита, defend
\newcommand*\dabcd{&{\small\ding{48}$\!\!^{12}_{34}$}} % защита, defend
\newcommand*\ac{&{\small\ding{48}$\!\!^1_{23}$}} % защита, defend
\newcommand*\db{&{\small\ding{48}$\!\!_2$}} % защита, defend
\newcommand*\dc{&{\small\ding{48}$\!\!_3$}} % защита, defend
\newcommand*\dd{&{\small\ding{48}$\!\!_4$}} % защита, defend
\newcommand*\bd{&{\small${}^2\!\!$\ding{48}$\!\!^3_{4}$}} % защита, defend
\newcommand*\de{&{\small\ding{48}$\!\!_5$}} % защита, defend
\newcommand*\dE{&{\small${}^4\!\!\!$\ding{48}$\!\!_5$}} % защита, defend
\newcommand*\cd{&{\small\ding{48}$\!\!^3_4$}} % защита, defend
\newcommand*\dg{&{\small\ding{48}$\!\!_6$}} % защита, defend
\newcommand*\fg{&{\small${}^6\!\!$\ding{48}$\!\!_7$}} % защита, defend
\newcommand*\dH{&{\small\ding{48}$\!\!_8$}} % защита, defend
\newcommand*\gh{&{\small\ding{48}$\!\!^7_8$}} % защита, defend
\newcommand*\fh{&{\small\ding{48}$\!\!^7_{89}$}} % защита, defend 
\newcommand*\ce{&{\small${}^3\!\!$\ding{48}$\!\!_5$}} % защита, defend
\newcommand*\ef{&{\small${}^5\!\!$\ding{48}$\!\!_6$}} % защита, defend
%\newcommand*\dh{&{\small\ding{48}$\!\!_8$}} % защита, defend
\newcommand*\di{&{\small\ding{48}$\!\!_9$}} % защита, defend
\newcommand*\cdef{&{\small ${}^2_4\!\!$\ding{48}$\!\!^{3}_{5}$}} % защита, defend
\newcommand*\cde{&{\small ${}^2\!\!$\ding{48}$\!\!^{3}_{5}$}} % защита, defend
\newcommand*\efg{&{\small ${}^5\!\!$\ding{48}$\!\!^{6}_{7}$}} % защита, defend
\newcommand*\befgh{&{\small ${}_2^5\!\!$\ding{48}$\!\!^{6}_{78}$}} % защита, defend
\newcommand*\Dh{&{\small${}^4\!\!$\ding{48}$\!\!_8$}} % защита, defend
\newcommand*\cfg{&{\small ${}^3\!\!$\ding{48}$\!\!^{6}_{7}$}} % защита, defend
\newcommand*\fgh{&{\small ${}^6\!\!$\ding{48}$\!\!^{7}_{8}$}} % защита, defend
\newcommand*\bce{&{\small ${}^2\!\!$\ding{48}$\!\!^{3}_{5}$}} % защита, defend
\newcommand*\dO{&{\small\ding{48}$\!\!_{15}$}}
\newcommand*\Skip{\noindent\rule{0.3cm}{0.9pt}}


\begin{document}
%\thispagestyle{empty}
% or
\pagenumbering{gobble}
%\AdvanceDate[-1] % печатаю в субботу а нужна пятница
%\begin{center}\today\end{center} % не до даты с 30 студентами
\vspace*{1\baselineskip} %landscape
\vspace{-0.9cm}
%\begin{table} \centering 
\newcommand*{\CS}{9pt} % ширина колонки
\begin{tabular}{p{7pt}|l|p{\CS}|p{\CS}|p{\CS}|p{\CS}|p{\CS}|p{\CS}|p{\CS}|p{\CS}|p{\CS}|p{\CS}}
%\multicolumn{16}{c}{График выполнения лабораторных работ студентами 8871 группы} \\ 
\multicolumn{11}{c}{Ведомость посещения занятий по датчикам студентами 9493 группы} \\
\toprule 
&&&&&&&&&&\\
&&&&&&&&&&\\
&&&&&&&&&&\\
&&&&&&&&&&\\
&&&&&&&&&&\\
&&&&&&&&&&\\
&&\rotatebox{90}{\rlap{\small 4 сентября}}
&\rotatebox{90}{\rlap{\small 18 сентября}}
&\rotatebox{90}{\rlap{\small 2 октября }}
&\rotatebox{90}{\rlap{\small 16 октября }}
&\rotatebox{90}{\rlap{\small 30 октября }}
&\rotatebox{90}{\rlap{\small 13 ноября }}
&\rotatebox{90}{\rlap{\small 27 ноября }}
&\rotatebox{90}{\rlap{\small 11 декабряя }}
&\rotatebox{90}{\rlap{\small 25 декабря }}
%&\rotatebox{90}{\rlap{\small 24 апреля }}
%&\rotatebox{90}{\rlap{\small 15 мая}}
%&\rotatebox{90}{\rlap{\small 29 мая }} 
\\
% commands vi to copy/paste D :+19 ->>> p :-18 :w
\midrule
1\,& Аверкиева Светлана Алексеевна   \ok\ok\ok\ok\ok\\
2\,& Большова Екатерина Алексеевна   \ok\no\ok\ok\ok\\
3\,& Варанкин Дмитрий Константинович \ok\ok\ok\ok\no\\
4\,& Варзопов Сергей Эдуардович      \ok\ok\ok\ok\ok\\
5\,& Викторов Артем Дмитриевич       \ok\no\ok\no\no\\
\midrule                                          
6\,& Зиняков Даниил                  \ok\ok\ok\ok\ok\\
7\,& Каменко Денис Валерьевич        \ok\ok\ok\ok\ok\\
8\,& Керимов Мовcар Мухтарович       \ok\no\ok\no\no\\
9\,& Курбатов Владимир Игоревич      \ok\ok\ok\no\ok\\
10\,& Мазулевский Михаил Даниилович  \ok\ok\no\no\ok\\
\midrule                                          
11\,& Мисуно Владислав Юрьевич       \ok\no\ok\ok\no\\
12\,& Нагин Артём Игоревич           \no\no\no\no\ok\\
13\,& Подопригора Вадим Евгеньевич   \ok\ok\ok\no\no\\
14\,& Починков Николай Дмитриевич    \ok\no\ok\no\ok\\
15\,& Пустозёрова Василиса           \no\ok\ok\ok\no\\
\midrule
16\,& Саппо Александр Алексеевич     \ok\ok\ok\ok\ok\\
17\,& Сенин Семён Павлович           \no\ok\no\no\ok\\
18\,& Сигаев Виталий Сергеевич       \ok\no\ok\ok\ok\\
19\,& Фан Фыок Минь Хоанг            \no\ok\no\ok\ok\\
20\,& Чернов Даниил Сергеевич        \ok\no\no\no\ok\\
\midrule
21\,& Шишкина Карина Валерьевна      \ok\ok\ok\ok\ok\\
22\,& Ярошук Владислав Андреевич     \ok\ok\ok\ok\ok\\
\bottomrule
\end{tabular} 
16 октября не работали lk, vec, mail
\newpage
%


\newpage
%
\begin{tabular}{l|llccccccccccccc}
\multicolumn{10}{c}{выполнение лабораторнах работ по датчикам, 9493 группа} \\
\toprule
&&Л1&Л1& Л2&Л2& Л3&Л3& Л4&Л4 &Л5&Л5& Л6&Л6\\
\midrule
1\,& Аверкиева Светлана Алексеевна   && && && &29.10& \\
2\,& Большова Екатерина Алексеевна   &&&&&&29.10&\\
3\,& Варанкин Дмитрий Константинович &17.09&18.09& 1.10&2.10\\
4\,& Варзопов Сергей Эдуардович      &24.08&2.10&12.10&16.10&     &     &22.10&\\
5\,& Викторов Артем Дмитриевич       \\
\midrule
6\,& Зиняков Даниил                  && && && &30.10&\\
7\,& Каменко Денис Валерьевич        &17.09&18.09&& && &30.10&\\
8\,& Керимов Мовcар Мухтарович       \\
9\,& Курбатов Владимир Игоревич      \\
10\,& Мазулевский Михаил Даниилович  \\
\midrule
11\,& Мисуно Владислав Юрьевич       \\
12\,& Нагин Артём Игоревич           \\
13\,& Подопригора Вадим Евгеньевич   \\
14\,& Починков Николай Дмитриевич    && && && &29.10&\\
15\,& Пустозёрова Василиса           \\
\midrule
16\,& Саппо Александр Алексеевич     && && && &30.10&\\
17\,& Сенин Семён Павлович           && && && &30.10&\\
18\,& Сигаев Виталий Сергеевич       && && && &23.10&\\
19\,& Фан Фыок Минь Хоанг            \\
20\,& Чернов Даниил Сергеевич        && && && &29.10&\\
\midrule
21\,& Шишкина Карина Валерьевна      && && && &30.10&  \\
22\,& Ярошук Владислав Андреевич     && && && &26.10\\
\bottomrule
\end{tabular}

\subsection*{Операционный усилитель}
\begin{itemize}
\item Каменко Денис Валерьевич -- в реальном операционном усилителе поставить правильное питание ОУ
\item Варзопов --
\end{itemize}

\subsection*{Мзмерение напряжения сети}
\begin{itemize}
\item Варзопов В промышленной синусоидальной сети напряжение дается действующим значением,
а в преобразованном по амплитуде нужно войти в заданный диапазон. Соответственно коэффициент трансформатора должен преобразовывать Uсети*sqrt(2) к Uсхемы 
В основной надписи нет проверяющего, еще одного студента, кто предварительно проверил работу. 
\item Сигаев -- тоже что и у Варзопова
\item Ярошук -- диоды не по ГОСТ (светлый треугольник с пересекающей линией от анода к катоду),
\item  Большова -- В промышленной синусоидальной сети напряжение дается действующим значением,
а в преобразованном по амплитуде нужно войти в заданный диапазон. Соответственно нужно преобразовывать Uсети*sqrt(2) к Uсхемы
\item Чернов -- нет цифрового следа: заголовка, темы, автора, ключевых слов. На выходе напряжение от 1.50в до 1.51в, т.е. не исполозован весь диапазон выходного напряжения от 0 до 3.3В
\end{itemize}
\end{document}
